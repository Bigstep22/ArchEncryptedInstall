\section{Installing Needed Command Line Applications}
\subsection{Basic Internet Installation}
For the internet connection manager I will be installing NetworkManager because it supports both wired and wireless connection management as well as having many possible gui applications that can be run on top of it to make its configuration easier.

Fore more information visit: \url{https://wiki.archlinux.org/index.php/NetworkManager}

To install it:
\begin{lstlisting}{language=bash}
$ sudo pacman -S networkmanager
\end{lstlisting}
Then enable its system service/daemon:
\begin{lstlisting}{language=bash}
$ sudo systemctl enable --now NetworkManager.service
\end{lstlisting}
This command enables the NetworkManager service while the --now flag starts it without having to invoke a second command or reboot to do so.
After starting the NetworkManager service it is possible to see the network status by executing :
\begin{lstlisting}{language=bash}
$ nmcli
\end{lstlisting}
And it is possible to start a graphical interface from the command line to manually create new connections (YES! Even Wifi connections as long as you know its interface \ref{sec:Internet_Connection}) by executing:
\begin{lstlisting}{language=bash}
$ nmtui
\end{lstlisting}

\subsection{Clone Dotfile Repository}
I will be cloning my personal dotfile repository in order to import all my settings easily:
\begin{lstlisting}{language=bash}
$ git clone https://link.to/repository
\end{lstlisting}
In order to easily install all of the dotfiles I will need GNU stow:
\begin{lstlisting}{language=bash}
$ sudo pacman -S stow
\end{lstlisting}
Now you can unstow any of the packages within the dotfile repo:
\begin{lstlisting}{language=bash}
$ stow -v -S folder_name
\end{lstlisting}
The -v flag enables more verbose output and -S instructs stow to actually stow the folder specified.


\subsection{Install Zsh and 'thefuck' Spell Correction and OpenSSH}
Now that all of the dotfiles are present for zsh and thefuck we can go ahead and install thenm:
\begin{lstlisting}{language=bash}
$ sudo pacman -S zsh thefuck openssh
\end{lstlisting}
Installing thefuck also takes care of installing Python since it is a dependency.
We also need to install openssh as zsh uses it in my default zsh config (only applies when using dotfiles)


\subsection{Install and Setup Command Line Tools}
Next install a large group of needed command line utilities that are useful during development and monitoring systems:
\begin{lstlisting}{language=bash}
$ sudo pacman -S tldr entr htop tig sl tmux ranger borg lshw efibootmgr
\end{lstlisting}
If desired, a program that can upload files to google drive can also be installed:
\begin{lstlisting}{language=bash}
$ trizen -S drive-bin
\end{lstlisting}
As well as removing a package that I personally don't need or want on my system:
\begin{lstlisting}{language=bash}
$ sudo pacman -Rnscu nano
\end{lstlisting}
NOTE: The '-Rnscu' flag recursively remove packages both that are not needed and packages that dependent on the package in question. If you execute this command with this flag on the wrong package you could brick or remove large chunks of your system BE CAREFUL.


