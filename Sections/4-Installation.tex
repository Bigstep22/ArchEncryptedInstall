\section{Installation}
\subsection{Initial Setup}
Start by selecting the proper mirror servers from /etc/pacman.d/mirrorlist by moving nearby server to the top in order to speedup downloads for this and future installations:
\begin{lstlisting}{language=bash}
# vim /etc/pacman.d/mirrorlist
\end{lstlisting}
\subsection{Installation}
Run lsblk one more time to make sure that all partitions are mounted properly.
\begin{lstlisting}{language=bash}
# lsblk
\end{lstlisting}
The moment we've all been waiting for, installing base Arch (as well as vim) onto the system:
\begin{lstlisting}{language=bash}
# pacstrap /mnt base vim
\end{lstlisting}

\subsection{fstab}
Since we have a custom partition setup the fstab setup is a little bit more involved than for the usual installation run both of the following:
\begin{lstlisting}{language=bash}
# genfstab -U /mnt >> /mnt/etc/fstab
# genfstab -L /mnt >> /mnt/etc/fstab
\end{lstlisting}
Then execute
\begin{lstlisting}{language=bash}
# vim /mnt/etc/fstab
\end{lstlisting}
And use vim to delete the three sets of lines referencing the lvm-based partitions by UUID and delete the one line referencing the boot partition by label.

An example fstab file is shown below as specified above:
\inputminted[linenos, fontsize=\small, baselinestretch=0.875, frame=lines]{shell}{Sections/4/fstab.txt}

\subsection{Chroot and Timezone}
Chroot into the new system:
\begin{lstlisting}{language=bash}
# arch-chroot /mnt
\end{lstlisting}
Set the timezone:
\begin{lstlisting}{language=bash}
# ln -sf /usr/share/zoneinfo/Region/City /etc/localtime
\end{lstlisting}
Which in the case of Amsterdam is:
\begin{lstlisting}{language=bash}
# ln -sf /usr/share/zoneinfo/Europe/Amsterdam /etc/localtime
\end{lstlisting}
Then run hwclock to generate /etc/adjtime:
\begin{lstlisting}{language=bash}
# hwclock --systohc
\end{lstlisting}
(This assumes the hardware clock is set to UTC)

\subsection{Localization and Network}
Uncomment en\textunderscore US.UTF-8 UTF-8 and other needed locales in /etc/locale.gen then generate the locales with:
\begin{lstlisting}{language=bash}
# locale-gen
\end{lstlisting}
Then set the LANG variable in locale.conf:
\begin{minted}[frame=lines]{bash}
/etc/locale.conf
======================================================================
LANG=en_US.UTF-8
\end{minted}
Create the hostname file:
\begin{minted}[frame=lines]{bash}
/etc/hostname
======================================================================
myhostname
\end{minted}
And corresponding hosts to the hosts file:
\begin{minted}[frame=lines]{bash}
/etc/hosts
======================================================================
127.0.0.1   localhost
::1         localhost
127.0.1.1   myhostname.localdomain	myhostname
\end{minted}
