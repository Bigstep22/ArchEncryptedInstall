
\section{Disk Setup}
This is where this document will deviate from the installation tutorial from the ArchWiki. This guide will attempt to document the installation of an  \href{https://wiki.archlinux.org/index.php/Dm-crypt/Encrypting_an_entire_system#LVM_on_LUKS}{LVM on LUKS encrypted system with an on-disk boot partition as shown here}.

\subsection{Partitioning}
Start by running lsblk to see your partition layout:
\begin{lstlisting}{language=bash}
# lsblk
\end{lstlisting}
Select the partition you want to use from the output (eg. sda) and run the following command for it:
\begin{lstlisting}{language=bash}
# gdisk /dev/sda
\end{lstlisting}
From there use the ? command to see a list of gdisk commands and figure out all of the needed commands.
Start by making the first partition (using the 'n' command) within gdisk of size ~512M for /boot of type EF00 (which should be the default type).

Then fill the rest with a second partition of type 8E00 (which will be our encrypted LVM volume) then use the w command within gdisk to write all the GTP data to the disk and exit.

\subsection{The Encryption}
Create the LUKS encrypted container using the command:
\begin{lstlisting}{language=bash}
# cryptsetup luksFormat --type luks2 /dev/sda2
\end{lstlisting}
After filling out your password twice, the encrypted partition should now be set up.
(Note: /dev/sda2 refers to the partition which will be the 'system' partition)

Now mount the encrypted partition so that it can be set up with the LVM and all needed partitions:
\begin{lstlisting}{language=bash}
# cryptsetup open /dev/sda2 crypt
\end{lstlisting}
Now the decrypted container should be available at /dev/mapper/crypt

\subsection{The Logical Volumes}
Start by creating a physical volume on top of the LUKS container:
\begin{lstlisting}{language=bash}
# pvcreate /dev/mapper/crypt
\end{lstlisting}
Followed by the creation of the volume group with the name 'volume':
\begin{lstlisting}{language=bash}
# vgcreate volume /dev/mapper/crypt
\end{lstlisting}
Now create all the needed volumes (subpartitions if you will) for the system:
\begin{lstlisting}{language=bash}
# lvcreate -L 8G volume -n swap
# lvcreate -L 32G volume -n root
# lvcreate -l 100%FREE volume -n home
\end{lstlisting}
You can modify the sizes '8G' and '32G' for whatever desired size the partitions will be.

\subsection{Formatting the Filesystems}
Format the filesystems for all the respective partitions on the system volumes:
\begin{lstlisting}{language=bash}
# mkswap /dev/volume/swap
# mkfs.ext4 /dev/volume/root
# mkfs.ext4 /dev/volume/home
\end{lstlisting}
After formatting the filesystems, mount them:
\begin{lstlisting}{language=bash}
# swapon /dev/volume/swap
# mount /dev/volume/root /mnt
# mkdir /mnt/home
# mount /dev/volume/home /mnt/home
\end{lstlisting}

\subsection{Boot Partition}
Since we don't have to do any fancy magic for the boot partition we can start right away by formatting (where sda1 is the desired boot partition):
\begin{lstlisting}{language=bash}
# mkfs.fat -F32 /dev/sda1
\end{lstlisting}
Followed by making the needed directory:
\begin{lstlisting}{language=bash}
# mkdir /mnt/boot
\end{lstlisting}
Followed by mounting:
\begin{lstlisting}{language=bash}
# mount /dev/sda1 /mnt/boot
\end{lstlisting}