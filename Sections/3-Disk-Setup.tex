
\section{Disk Setup}
This is where this document will deviate from the installation tutorial from the ArchWiki. This guide will attempt to document the installation of an  \href{https://wiki.archlinux.org/index.php/Dm-crypt/Specialties#Encrypted_system_using_a_detached_LUKS_header}{LVM on LUKS encrypted system using aa detached LUKS header}.

\subsection{Encrypting The Disk}
Start by running lsblk to see your partition layout:
\begin{lstlisting}{language=bash}
# lsblk
\end{lstlisting}
First create an empty header file
\begin{lstlisting}{language=bash}
# dd if=/dev/zero of=header.img bs=4M count=1
\end{lstlisting}
Select the block device you want to use from the output (eg. sda) and run the following command for it to create the LUKS encrypted container using the command:

\begin{lstlisting}{language=bash}
# cryptsetup luksFormat --type luks2 /dev/sda --align-payload 8192 --header header.img
\end{lstlisting}
After filling out your password twice, the encrypted partition should now be set up.
(Note: /dev/sda refers to the block device which will be the encrypted 'system' partition)

From the articel linked above: (Tip: When using option --header, --align-payload allows specifying the start of encrypted data on a device. By reserving a space at the beginning of device you have the option of later reattaching the LUKS header. See cryptsetup(8) for values supported by --align-payload.)

Now mount the encrypted partition so that it can be set up with the LVM and all needed partitions:
\begin{lstlisting}{language=bash}
# cryptsetup open --header header.img /dev/sda crypt
\end{lstlisting}
Now the decrypted container should be available at /dev/mapper/crypt

\subsection{The Logical Volumes}
Start by creating a physical volume on top of the LUKS container:
\begin{lstlisting}{language=bash}
# pvcreate /dev/mapper/crypt
\end{lstlisting}
Followed by the creation of the volume group with the name 'volume':
\begin{lstlisting}{language=bash}
# vgcreate volume /dev/mapper/crypt
\end{lstlisting}
Now create all the needed volumes (subpartitions if you will) for the system:
\begin{lstlisting}{language=bash}
# lvcreate -L 8G volume -n swap
# lvcreate -L 32G volume -n root
# lvcreate -l 100%FREE volume -n home
\end{lstlisting}
You can modify the sizes '8G' and '32G' for whatever desired size the partitions will be.

\subsection{Formatting the Filesystems}
Format the filesystems for all the respective partitions on the system volumes:
\begin{lstlisting}{language=bash}
# mkswap /dev/volume/swap
# mkfs.ext4 /dev/volume/root
# mkfs.ext4 /dev/volume/home
\end{lstlisting}
After formatting the filesystems, mount them:
\begin{lstlisting}{language=bash}
# swapon /dev/volume/swap
# mount /dev/volume/root /mnt
# mkdir /mnt/home
# mount /dev/volume/home /mnt/home
\end{lstlisting}

\subsection{Boot Partition}
Since we will be using a seperate device for the boot partition we need to partition that disk and then formatting it (where /dev/sdb is the device and sdb1 is the desired boot partition):
\begin{lstlisting}{language=bash}
# gdisk /dev/sdb
\end{lstlisting}
From there use the ? command to see a list of gdisk commands and figure out all of the needed commands.

Start by making the boot first partition (using the 'n' command) within gdisk of size ~512M(or larger if you want to only use the USB key as a boot key) for /boot of type EF00 (which should be the default type).

Then fill the rest of the drive however you like (although I don't recommend using the usb key for anything other than as a boot drive since the loss of the USB key could mean loss of the entire installed system)
Format the boot partition such that it can be used on UEFI:
\begin{lstlisting}{language=bash}
# mkfs.fat -F32 /dev/sdb1
\end{lstlisting}
Followed by making the needed directory:
\begin{lstlisting}{language=bash}
# mkdir /mnt/boot
\end{lstlisting}
Followed by mounting:
\begin{lstlisting}{language=bash}
# mount /dev/sdb1 /mnt/boot
\end{lstlisting}
And copyint the header file to the boot (USB) partition:
\begin{lstlisting}{language=bash}
# cp header.img /mnt/boot
\end{lstlisting}
